\documentclass[12pt]{article}
\usepackage{paralist,amsmath,amssymb,thumbpdf,lmodern}
\usepackage{algorithm}
\usepackage{algpseudocode}
\usepackage[margin=0.8in]{geometry}
\usepackage{graphicx}
\usepackage{enumerate}
\usepackage{physics}
\usepackage{amsfonts}
\usepackage{bbm}
\usepackage{xcolor}
\usepackage{subcaption}
\usepackage{graphicx}
\usepackage{multirow}
\usepackage{amsfonts}
\usepackage{booktabs}
\usepackage{diagbox}
\usepackage[outdir=./]{epstopdf}
\usepackage{caption}
\usepackage{bm}
\usepackage{makecell}
\usepackage{ifxetex,ifluatex}
\newcommand*\diff{\mathop{}\!\mathrm{d}}
\begingroup\expandafter\expandafter\expandafter\endgroup
\expandafter\ifx\csname IncludeInRelease\endcsname\relax
  \usepackage{fixltx2e}
\fi
\ifnum 0\ifxetex 1\fi\ifluatex 1\fi=0 % if pdftex
  \usepackage[T1]{fontenc}
  \usepackage[utf8]{inputenc}
  \usepackage{textcomp} % provides euro and other symbols
\else % if luatex or xelatex
  \usepackage{unicode-math}
  \defaultfontfeatures{Ligatures=TeX,Scale=MatchLowercase}
\fi
\usepackage{hyperref}
\hypersetup{
    colorlinks=true,
    linkcolor=blue,
    filecolor=magenta,      
    urlcolor=blue,
}
\urlstyle{same}  % don't use monospace font for urls
\setlength{\emergencystretch}{3em}  % prevent overfull lines
\usepackage{newtxtext,newtxmath}

\title{Biostat 830 Midterm}
\date{}
\author{Mukai Wang 98830336}

\makeatletter
\newenvironment{breakablealgorithm}
{% \begin{breakablealgorithm}
		\begin{center}
			\refstepcounter{algorithm}% New algorithm
			\hrule height.8pt depth0pt \kern2pt% \@fs@pre for \@fs@ruled
			\renewcommand{\caption}[2][\relax]{% Make a new \caption
				{\raggedright\textbf{\fname@algorithm~\thealgorithm} ##2\par}%
				\ifx\relax##1\relax % #1 is \relax
				\addcontentsline{loa}{algorithm}{\protect\numberline{\thealgorithm}##2}%
				\else % #1 is not \relax
				\addcontentsline{loa}{algorithm}{\protect\numberline{\thealgorithm}##1}%
				\fi
				\kern2pt\hrule\kern2pt
			}
		}{% \end{breakablealgorithm}
		\kern2pt\hrule\relax% \@fs@post for \@fs@ruled
	\end{center}
}
\makeatother
\DeclareMathOperator*{\argmax}{arg\,max}
\DeclareMathOperator*{\argmin}{arg\,min}


\begin{document}
\maketitle

\section*{Problem 1}

 

\section*{Problem 2}


\begin{enumerate}[(a)]


\item The SEIR model is

\begin{align*}
	\frac{\diff S}{\diff t} &= -\beta S \frac{I}{N}\\
	\frac{\diff E}{\diff t} &= \beta S \frac{I}{N} - \delta E\\
	\frac{\diff I}{\diff t} &= \delta E - \gamma I\\
	\frac{\diff R}{\diff t} &= \gamma I
\end{align*}

in which $N=S+E+I+R$. According to the problem setting, $N=1000000$. On day zero, $S_{0}=999950$, $E_{0}=0$, $I_{0}=50$ and $R_{0}=0$. Because an individual who contracts th disease takes on average 3 days to become infectious, $\delta=1/3$ $\text{day}^{-1}$. Because an individual who is infectious takes on average 2 weeks to recover, $\gamma=1/14$ $\text{day}^{-1}$. Because the true reproduction number $R_{0}=10$, the true $\beta=R_{0} \gamma=5/7$ $\text{day}^{-1}$.

When estimating the unknown $\beta$, I propose \emph{least squares} estimation. I define the search space for $\beta$ to range from 0.01 to 1. For each candidate value, I calculate the population structure trajectory. The least square estimator $\hat{\beta}_\text{ls}$ is the one that minimizes the sum of squares of errors between predicted $\hat{\bm{I}}$ trajectory and the observed $\bm{I}$ trajectory. The details are in Algorithm \ref{LS}.

\begin{breakablealgorithm}
	\caption{Least Squares Algorithm for Estimating Transmission Rate $\beta$ in SEIR Model}\label{LS}
	\hspace*{\algorithmicindent} \textbf{Input} Number of follow up days $T$,  Population trajectory in \emph{susceptible} compartment $\bm{S}$, population trajectory in \emph{exposed} compartment $\bm{E}$, population trajectory in \emph{infectious} compartment $\bm{I}$, population trajectory in \emph{recovery} compartment $\bm{R}$, known transmission rates $\gamma$ and $\delta$
	
	\begin{algorithmic}[1]
		\For{$i \gets 1$ to 100}
		\State $\hat{\beta}_{i} \gets i/100$ \Comment{search for parameter values from 0.01 to 1}
		\State $\hat{S}_{i,0} \gets S_{0}$ \Comment{Set initial values from observed data}
		\State $\hat{E}_{i,0} \gets E_0$
		\State $\hat{I}_{i,0} \gets I_{0}$
		\State $\hat{R}_{i,0} \gets R_0$
			\For{$j \gets 1$ to $T$}
				\State Calculate $\hat{S}_{i,j}$, $\hat{E}_{i,j}$, $\hat{I}_{i,j}$, $\hat{R}_{i,j}$ from $\hat{S}_{i,j-1}$, $\hat{E}_{i,j-1}$, $\hat{I}_{i,j-1}$, $\hat{R}_{i,j-1}$ using Runge-Kutta algorithm.
			\EndFor
		\State $\text{SSE}_{i} \gets \sum_{j=1}^{T} (\hat{I}_{i,j} - I_{j})^2$
		\EndFor
		\State $\hat{\beta}_\text{ls} \gets \frac{1}{100}\underset{i}{\mathrm{argmin}}\text{SSE}_{i}$	
	\end{algorithmic}
	
	\hspace*{\algorithmicindent} \textbf{Output} $\hat{\beta}_\text{ls}$
\end{breakablealgorithm}

	\item Following the algorithm specified in the previous question, I found out that $\hat{\beta}_\text{ls} = 0.71$. The estimation bias is $0.71 - 5/7 = -0.004$. The SSE plot against all $\beta$ values is in Figure \ref{p2}A.

	
	
	\item To account for 10\% underreporting in the infectious compartment, I adjust the coefficients in the SEIR model
	\begin{align*}
	\frac{\diff S}{\diff t} &= -\frac{10}{9} \beta S \frac{I}{N}\\
	\frac{\diff E}{\diff t} &= \frac{10}{9}\beta S \frac{I}{N} - \delta E\\
	\frac{\diff I}{\diff t} &= \frac{9}{10}\delta E - \gamma I\\
	\frac{\diff R}{\diff t} &= \frac{1}{10}\delta E + \gamma I
	\end{align*}
	
	In this case $\hat{\beta}_\text{ls} = 0.70$. The estimation bias is $0.70 - 5/7 = -0.014$, which is almost the same as the scenario without underreporting. The slight adjustment in the SEIR model can adapt well to minor underreporting. The SSE plot against all $\beta$ values is in Figure \ref{p2}B.
	
	\begin{figure}[htbp]
		\centering
		\includegraphics[scale=0.7]{P2plot.pdf}
		\caption{The sum of squares of error of the infectious compartment trajectory for $\beta$ ranging from 0.01 to 1. Panel A corresponds to the scenario with no underreporting. Panel B corresponds to the scenario with 10\% underreporting in the infectious compartment.}\label{p2}
	\end{figure}
\end{enumerate}




\section*{Problem 3}





\end{document}
