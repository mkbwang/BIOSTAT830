\documentclass[12pt]{article}
\usepackage{paralist,amsmath,amssymb,thumbpdf,lmodern}
\usepackage{algorithm}
\usepackage{algpseudocode}
\usepackage[margin=0.8in]{geometry}
\usepackage{graphicx}
\usepackage{enumerate}
\usepackage{physics}
\usepackage{amsfonts}
\usepackage{bbm}
\usepackage{xcolor}
\usepackage{subcaption}
\usepackage{graphicx}
\usepackage{multirow}
\usepackage{amsfonts}
\usepackage{booktabs}
\usepackage{diagbox}
\usepackage[outdir=./]{epstopdf}
\usepackage{caption}
\usepackage{bm}
\usepackage{makecell}
\usepackage{ifxetex,ifluatex}
\newcommand*\diff{\mathop{}\!\mathrm{d}}
\begingroup\expandafter\expandafter\expandafter\endgroup
\expandafter\ifx\csname IncludeInRelease\endcsname\relax
  \usepackage{fixltx2e}
\fi
\ifnum 0\ifxetex 1\fi\ifluatex 1\fi=0 % if pdftex
  \usepackage[T1]{fontenc}
  \usepackage[utf8]{inputenc}
  \usepackage{textcomp} % provides euro and other symbols
\else % if luatex or xelatex
  \usepackage{unicode-math}
  \defaultfontfeatures{Ligatures=TeX,Scale=MatchLowercase}
\fi
\usepackage{hyperref}
\hypersetup{
    colorlinks=true,
    linkcolor=blue,
    filecolor=magenta,      
    urlcolor=blue,
}
\urlstyle{same}  % don't use monospace font for urls
\setlength{\emergencystretch}{3em}  % prevent overfull lines
\usepackage{newtxtext,newtxmath}

\title{Biostat 830 HW2}
\date{}
\author{Mukai Wang 98830336}

\makeatletter
\newenvironment{breakablealgorithm}
{% \begin{breakablealgorithm}
		\begin{center}
			\refstepcounter{algorithm}% New algorithm
			\hrule height.8pt depth0pt \kern2pt% \@fs@pre for \@fs@ruled
			\renewcommand{\caption}[2][\relax]{% Make a new \caption
				{\raggedright\textbf{\fname@algorithm~\thealgorithm} ##2\par}%
				\ifx\relax##1\relax % #1 is \relax
				\addcontentsline{loa}{algorithm}{\protect\numberline{\thealgorithm}##2}%
				\else % #1 is not \relax
				\addcontentsline{loa}{algorithm}{\protect\numberline{\thealgorithm}##1}%
				\fi
				\kern2pt\hrule\kern2pt
			}
		}{% \end{breakablealgorithm}
		\kern2pt\hrule\relax% \@fs@post for \@fs@ruled
	\end{center}
}
\makeatother


\begin{document}
\maketitle

\section*{Problem 1}

\begin{enumerate}[(a)]
	\item Since $Y\sim \text{Poisson}(\mu)$ and $\mu \sim \text{Gamma}(\kappa, \theta)$, 
	\begin{align*}
	P\left(\left.Y \right\vert\mu \right)& = \frac{e^{-\mu} \mu^{Y}}{Y!}\\
	\pi\left(\left.\mu \right\vert \kappa, \theta\right) &= \frac{\theta^{\kappa}}{\Gamma(\kappa)}\mu^{\kappa - 1}e^{-\theta\mu}\diff \mu 
	\end{align*}
	Therefore the marginal distribution of $Y$ is
	\begin{align*}
		\pi\left(\left.Y\right\vert\kappa, \theta \right)&= \int_{0}^{\infty} P\left(\left.Y \right\vert\mu \right) \pi\left(\left.\mu \right\vert \kappa, \theta\right)\diff \mu\\
		&=\frac{\theta^{\kappa}}{Y!\cdot \Gamma(\kappa)}\cdot \int_{0}^{\infty}e^{-(1+\theta)\mu} \cdot \mu^{Y+K-1}\diff \mu \\
		&=\frac{\theta^{\kappa} \Gamma(Y+\kappa)}{Y! \cdot \Gamma(\kappa) \left(1+\theta\right)^{Y+\kappa} }\cdot \int_{0}^{\infty} \frac{(1+\theta)^{Y+\kappa}}{\Gamma(Y+\kappa)}e^{-(1+\theta)\mu}\mu^{Y+K-1}\diff \mu\\
		&=\begin{pmatrix}Y+\kappa - 1 \\ Y\end{pmatrix}\left(\frac{1}{1+\theta}\right)^{Y}\left(\frac{\theta}{1+\theta}\right)^{\kappa}
	\end{align*}
	If we consider $\kappa$ to be an integer, then we have $Y \sim \text{NegBin}(\kappa, 1/(1+\theta))$.
	\item Given that the marginal distribution is a negative binomial distribution, then we have the expectation and variance to be
	\begin{align*}
		\mathbb{E}[Y] &= \kappa\theta\\
		\text{Var}(Y) &= \kappa\theta(1+\theta)
	\end{align*}
	\item We can reparameterize the expression s of expectation and variance to be
	\begin{align*}
		\mathbb{E}[Y] &= \mu \\
		\text{Var}(Y) &= \mu(1+\sigma^2 \mu)
	\end{align*}
	in which $\mu = \kappa\theta$ and $\sigma^2 = 1/\kappa$. We can tell that the shape parameter of the gamma distribution dictates the overdispersion. A suitable null hypothesis is $H_{0} : \sigma^2 = 0$
\end{enumerate}
 

\section*{Problem 2}
\emph{Note: Because the provisional weekly COVID-19 deaths by age and sex are only available on \href{https://data.cdc.gov/NCHS/Provisional-COVID-19-Deaths-by-Week-Sex-and-Age/vsak-wrfu}{national level}, I also referred to the weekly aggregated death counts of Michigan from \href{https://www.michigan.gov/coronavirus/-/media/Project/Websites/coronavirus/Michigan-Data/10-04-2022/Datasets/Cases-and-Deaths-by-County-and-by-Date-of-Symptom-Onset-or-by-Date-of-Death2022-10-04.xlsx?rev=d9568cd19a45423aac6c03af0ce87707&hash=EF2BBF61BD1BFFB0E9F20527E91DCC47}{the state government}. The state level death counts by age groups and sex groups are estimated based on state level totals and national level proportions.}  
\begin{enumerate}[(a)]
	\item From the weekly death count time series plot(Figure \ref{weekdeath}), I can notice that death events peaked at four time points for people of all age groups and sex groups. They are April to May 2020(beginning of pandemic), December 2020 to January 2021(the first winter), May 2021(the second spring) and December 2021 to January 2022(the second winter). In panel A, I plot the death trends by three age groups(0-14, 15-64, >65). There were virtually no death cases for the youngest age group. Death counts from the oldest age group have been consistently higher than the other two age groups, and were considerably higher during the four severe periods. There hasn't been significant differences between male and female in terms of death cases throughout the past two years. The death counts of males were slightly higher than females during the four severe periods.
	\begin{figure}[htbp]
		\centering
		\includegraphics[scale=0.7]{DeathTrajectory.pdf}
		\caption{Weekly Death Count Trends by Age Group(A) and Sex Group(B)}\label{weekdeath}
	\end{figure} 
	
	\item The vaccination data is provided for four age groups(5-11, 12-17, 18-64, 65+). The emergency use authorization by CDC were given for these different age groups on different dates. Adults between 18-64 began vaccination on 2/13/2021. Elderly people older than 65 began vaccination on 3/5/2021. Adolescents between 12 and 18 began vaccination on 5/13/2021. Children between 5 and 12 began vaccination on 11/3/2021. According to Figure \ref{vaccinationplot}A, the proportion of population that had at least one dose rose quickly at the beginning and later plateaued for population of all ages. The oldest population had the highest one-dose vaccination rate, followed by adults, teenagers and children.
	
	\item According to Figure \ref{vaccinationplot}B, the trend for the fully vaccinated is the basically the same as those with at least one dose. The only difference is that the full vaccination rate is slightly lower than the one-dose vaccination rate by about 5\% for all age groups by the end of September 2022.
	
	\begin{figure}[htbp]
		\centering
		\includegraphics[scale=0.7]{vaccination.pdf}
		\caption{The proportion of population with at least one dose of vaccination(A) and the proportion of fully vaccinated population(B)}\label{vaccinationplot}
	\end{figure}
\end{enumerate}


\section*{Problem 3}

\begin{enumerate}[(a)]
	\item We notice that there are barely any death cases for children younger than 14, therefore we only compare people of age 15-64 and people older than 65. We first fit a Poisson regression on the weekly death counts across 132 weeks. The Poisson model with log link is parameterized as
	\begin{align*}
		\log(\frac{\mathbb{E}[n_{i0}]}{N_{0}}) &= \beta_{0} + \alpha t_{i}\\
		\log(\frac{\mathbb{E}[n_{i1}]}{N_{1}}) &= \beta_{0} + \alpha t_{i} + \beta_{1}
	\end{align*}
	$n_{i0}$ is the death count of people aged between 15 and 64 in week $t_{i}$. $n_{i1}$ is the death count of people older than 65. I scale the week number so that $t_{i} = i$ ($i=0,1,\cdots 131$). $N_{0}$ is the total population aged between 15 and 64 in Michigan. $N_{1}$ is the total population older than 65. According to \href{https://www.census.gov/quickfacts/MI}{census bureau data}, $N_{0}\approx 6.08$ million and $N_{1}\approx 1.82$ million. $\alpha$ represents the effect of time trend.  $\beta_1$ represents the difference in death rates between people older than 65 and people between 15-64 years old adjusted for time. $\beta_{1}$ is the parameter of interest.  $\hat{\beta}_{1} = 2.307$ and $\text{SE}(\hat{\beta}_{1}) = 0.012$. The p value was smaller than $2\times 10^{-16}$, indicating that age is a significant risk factor. The death rate of senior people is about $\exp(2.3) = 10$ times the death rate of people between 15 and 64. $\hat{\alpha} = -0.0026$ and $\text{SE}(\hat{\alpha})=1.3\times 10^{-4}$. The p value for $\hat{\alpha}$ is also smaller than $2\times 10^{-16}$, indicating that the death rate is steadily decreasing month by month, although the death rate is only decreasing at a tiny $1-\exp(-0.0025)=0.25\%$.
	\item In the previous question, $\text{deviance}/\text{df}=30277/261=116 > 1$, indicating that overdispersion exists in the data. We use negative binomial regression instead of Poisson regression, which means the variance $V(n) = \mathbb{E}[n] +\mathbb{E}^{2}[n]/\theta$. Under the new setting, $\hat{\beta}_{1} = 2.30$ and  $\text{SE}(\hat{\beta}_1) = 0.114$. The significance conclusion doesn't change but $\text{deviance}/\text{df}=300/261=1.15$ is much smaller than that of Poisson regression. $\hat{\theta}=1.18$. $\hat{\alpha} = -0.002$ and $\text{SE}(\hat{\alpha}) = 0.0014$, indicating that the time trend is not significant any more after fitting negative binomial regression to account for overdispersion.
	
	\item In this subquestion we set five different slopes for the time trends separated by four time knots. The first time knot is set at December 14, 2020(availability to the general public, week 81). The second time knot is set at June 26, 2021(the time when the percent of infection by the delta variant exceeded 50\% of the US population, week 137). The third time knot is set at Nov 29, 2021(the time when the omicron variant began to spread in South Africa, week 181). The fourth time knot is set at May 22, 2022(when the omicron became dominant in the USA, 231). The broken-stick model can be parametrized as
	 \begin{align*}
		\log(\frac{\mathbb{E}[n_{i0}]}{N_{0}}) &= \beta_{0} + \left(\alpha_{0} + \alpha_{1}\mathbb{I}(t_{i} \geq 81) + \alpha_{2}\mathbb{I}(t_{i} \geq 137) +\alpha_{3} \mathbb{I}(t_{i} \geq 181) + \alpha_{4}\mathbb{I}(t_{i} \geq 231)\right) t_{i}\\
		\log(\frac{\mathbb{E}[n_{i1}]}{N_{1}}) &= \beta_{0} + \left(\alpha_{0} + \alpha_{1}\mathbb{I}(t_{i} \geq 81) + \alpha_{2}\mathbb{I}(t_{i} \geq 137) +\alpha_{3} \mathbb{I}(t_{i} \geq 181) + \alpha_{4}\mathbb{I}(t_{i} \geq 231)\right) t_{i} + \beta_{1}
	\end{align*}
	The summary of the Poisson regression coefficients are in table \ref{stickpoisson}. The estimated difference of log death rate between two populations is the same as the previous questions. The death rates were decreasing throughout time, but the decrease rate of log death rate changed at each time knot that we chose. The decrease rate slowed down after Dec 14 2020, sped up after June 26, 2021, slowed down again after Nov 29, 2021 and sped up after May 22, 2022. 
	\begin{table}[htbp]
		\centering
		\begin{tabular}{ccccc}
			\toprule
			Coefficient & Effect Size & SE & P value & Meaning\\
			\midrule
			$\alpha_{0}$ & $-0.010$ & $7.6\times 10^{-4}$ & $<2\times 10^{-16}$ & \makecell{Average decrease per week of \\ log Death Rate of  Population \\ between 15 and 64 \\ at the beginning}\\
			\addlinespace[0.2cm]
			$\alpha_{1}$ & $0.0067$ & $5.7\times 10^{-4}$ & $<2\times 10^{-16}$& \makecell{Change in decrease rate \\ of log death rate\\ after Dec 14, 2020}\\
			\addlinespace[0.2cm]
			$\alpha_{2}$ & $-0.0018$ & $2.8\times 10^{-4}$ & $9.38\times 10^{-11}$& \makecell{Change in decrease rate \\ of log death rate\\ after June 26, 2021}\\
			\addlinespace[0.2cm]
			$\alpha_{3}$ & $0.0057$ & $2.04\times 10^{-4}$ & $<2\times 10^{-16}$& \makecell{Change in decrease rate \\ of log death rate\\ after Nov 29, 2021}\\
			\addlinespace[0.2cm]
			$\alpha_{4}$ & $-0.0095$ & $2.04\times 10^{-4}$ & $<2\times 10^{-16}$& \makecell{Change in decrease rate \\ of log death rate\\ after May 22, 2022}\\
			\addlinespace[0.2cm]
			$\beta_{1}$ & 2.31 & 0.012 & $<2\times 10^{-16}$ & \makecell{Difference of log death rate \\between population aged 65+ \\ and population aged between 15 and 65}\\
			\bottomrule
		\end{tabular}
		\caption{Poisson Regression with Five Broken Sticks for the time trend}\label{stickpoisson}
	\end{table}	
	
	
	The summary of the negative binomial regression coefficients are in table \ref{sticknb}. The big difference between negative binomial regression and poisson regression lies in the inference of the slopes of the "broken sticks". After considering overdispersion, the decrease rate of the log death rate stayed relatively unchanged until May 22 2022.
	
	
	\begin{table}[htbp]
		\centering
		\begin{tabular}{ccccc}
			\toprule
			Coefficient & Effect Size & SE & P value & Meaning\\
			\midrule
			$\alpha_{0}$ & $-0.0068$ & $0.008$ & $0.40$ & \makecell{Average decrease per week of \\ log Death Rate of  Population \\ between 15 and 64 \\ at the beginning}\\
			\addlinespace[0.2cm]
			$\alpha_{1}$ & $0.007$ & $0.006$ & $0.2435$& \makecell{Change in decrease rate \\ of log death rate\\ after Dec 14, 2020}\\
			\addlinespace[0.2cm]
			$\alpha_{2}$ & $-0.0011$ & $0.0029$ & $0.70$& \makecell{Change in decrease rate \\ of log death rate\\ after June 26, 2021}\\
			\addlinespace[0.2cm]
			$\alpha_{3}$ & $0.0036$ & $0.0021$ & $0.086$& \makecell{Change in decrease rate \\ of log death rate\\ after Nov 29, 2021}\\
			\addlinespace[0.2cm]
			$\alpha_{4}$ & $-0.011$ & $0.001$ & $3.1\times 10^{-11}$& \makecell{Change in decrease rate \\ of log death rate\\ after May 22, 2022}\\
			\addlinespace[0.2cm]
			$\beta_{1}$ & 2.38 & 0.107 & $<2\times 10^{-16}$ & \makecell{Difference of log death rate \\between population aged 65+ \\ and population aged between 15 and 65}\\
			\bottomrule
		\end{tabular}
		\caption{Negative Binomial Regression with Five Broken Sticks for the time trend}\label{sticknb}
	\end{table}
\end{enumerate}


\section*{Problem 4}

\begin{enumerate}[(a)]
	\item After adding the full vaccination information as the main effect, the model becomes
	\begin{align*}
		\log(\frac{\mathbb{E}[n_{i0}]}{N_{0}}) &= \beta_{0} + \left(\alpha_{0} + \alpha_{1}\mathbb{I}(t_{i} \geq 81) + \alpha_{2}\mathbb{I}(t_{i} \geq 137) +\alpha_{3} \mathbb{I}(t_{i} \geq 181) + \alpha_{4}\mathbb{I}(t_{i} \geq 231)\right) t_{i} + \gamma_{1} v_{i0}\\
		\log(\frac{\mathbb{E}[n_{i1}]}{N_{1}}) &= \beta_{0} + \left(\alpha_{0} + \alpha_{1}\mathbb{I}(t_{i} \geq 81) + \alpha_{2}\mathbb{I}(t_{i} \geq 137) +\alpha_{3} \mathbb{I}(t_{i} \geq 181) + \alpha_{4}\mathbb{I}(t_{i} \geq 231)\right) t_{i} + \beta_{1} + \gamma_{1} v_{i1}
	\end{align*}
	in which $\gamma_{1}$ is the effect size of the fully vaccination rate. $v_{i0}$ is the fully vaccination percentage of the population aged between 18 and 64 by the end of week $t_{i}$. $v_{i1}$ is the fully vaccination percentage of population older than 65 by the end of week $t_{i}$. I fit the negative binomial distribution to take into account overdispersion. The coefficients are summarized in Table \ref{sticknbfullvc}. The fully vaccination rate didn't significantly reduce the mortality rate. The other coefficients are almost the same as the negative binomial model without vaccination rate(Table \ref{sticknb}).
	
	
	\begin{table}[htbp]
		\centering
		\begin{tabular}{ccccc}
			\toprule
			Coefficient & Effect Size & SE & P value & Meaning\\
			\midrule
			$\alpha_{0}$ & $-0.0057$ & $0.008$ & $0.48$ & \makecell{Average decrease per week of \\ log Death Rate of  Population \\ between 15 and 64 \\ at the beginning}\\
			\addlinespace[0.2cm]
			$\alpha_{1}$ & $0.011$ & $0.006$ & $0.083$& \makecell{Change in decrease rate \\ of log death rate\\ after Dec 14, 2020}\\
			\addlinespace[0.2cm]
			$\alpha_{2}$ & $8.5\times 10^{-4}$ & $0.0031$ & $0.785$& \makecell{Change in decrease rate \\ of log death rate\\ after June 26, 2021}\\
			\addlinespace[0.2cm]
			$\alpha_{3}$ & $0.0030$ & $0.0022$ & $0.173$& \makecell{Change in decrease rate \\ of log death rate\\ after Nov 29, 2021}\\
			\addlinespace[0.2cm]
			$\alpha_{4}$ & $-0.013$ & $0.0018$ & $2.4\times 10^{-12}$& \makecell{Change in decrease rate \\ of log death rate\\ after May 22, 2022}\\
			\addlinespace[0.2cm]
			$\beta_{1}$ & 2.50 & 0.128 & $<2\times 10^{-16}$ & \makecell{Difference of log death rate \\between population aged 65+ \\ and population aged between 15 and 65}\\
			\addlinespace[0.2cm]
			$\gamma_{1}$ & $-0.0077$ & $0.0043$ & 0.071 & \makecell{Effect Size of \\fully vaccination percentage}\\
			\bottomrule
		\end{tabular}
		\caption{Negative Binomial Regression with Five Broken Sticks for the Time Trend and Full Vaccination Rate}\label{sticknbfullvc}
	\end{table}	
	
	\item We add an interaction term between age groups and vaccination to determine if there is significant difference between senior population and younger adults regarding effectiveness of full vaccination on reducing mortality. The model is
		\begin{align*}
		\log(\frac{\mathbb{E}[n_{i0}]}{N_{0}}) &= \beta_{0} + \left(\alpha_{0} + \alpha_{1}\mathbb{I}(t_{i} \geq 81) + \alpha_{2}\mathbb{I}(t_{i} \geq 137) +\alpha_{3} \mathbb{I}(t_{i} \geq 181) + \alpha_{4}\mathbb{I}(t_{i} \geq 231)\right) t_{i} + \gamma_{1} v_{i0}\\
		\log(\frac{\mathbb{E}[n_{i1}]}{N_{1}}) &= \beta_{0} + \left(\alpha_{0} + \alpha_{1}\mathbb{I}(t_{i} \geq 81) + \alpha_{2}\mathbb{I}(t_{i} \geq 137) +\alpha_{3} \mathbb{I}(t_{i} \geq 181) + \alpha_{4}\mathbb{I}(t_{i} \geq 231)\right) t_{i} +  \\
		&\quad \beta_{1} +(\gamma_{1}+\gamma_{2}) v_{i1}
	\end{align*}
	
	There is an extra parameter $\gamma_{2}$ to represent the difference in effect sizes of full vaccination rates between senior population and younger population. The fitted parameters are in table \ref{sticknbfullvcage}. There is no significant difference between 65+ population and 18-64 population in terms of effectiveness of full vaccination on reducing mortality rate.
	
	\begin{table}[htbp]
		\centering
		\begin{tabular}{ccccc}
			\toprule
			Coefficient & Effect Size & SE & P value & Meaning\\
			\midrule
			$\alpha_{0}$ & $-0.0064$ & $0.0082$ & $0.43$ & \makecell{Average decrease per week of \\ log Death Rate of  Population \\ between 15 and 64 \\ at the beginning}\\
			\addlinespace[0.2cm]
			$\alpha_{1}$ & $0.010$ & $0.006$ & $0.124$& \makecell{Change in decrease rate \\ of log death rate\\ after Dec 14, 2020}\\
			\addlinespace[0.2cm]
			$\alpha_{2}$ & $-2.4\times 10^{-4}$ & $0.0032$ & $0.942$& \makecell{Change in decrease rate \\ of log death rate\\ after June 26, 2021}\\
			\addlinespace[0.2cm]
			$\alpha_{3}$ & $0.0034$ & $0.0022$ & $0.132$& \makecell{Change in decrease rate \\ of log death rate\\ after Nov 29, 2021}\\
			\addlinespace[0.2cm]
			$\alpha_{4}$ & $-0.012$ & $0.0018$ & $5.07\times 10^{-11}$& \makecell{Change in decrease rate \\ of log death rate\\ after May 22, 2022}\\
			\addlinespace[0.2cm]
			$\beta_{1}$ & 2.61 & 0.164 & $<2\times 10^{-16}$ & \makecell{Difference of log death rate \\between population aged 65+ \\ and population aged between 15 and 65}\\
			\addlinespace[0.2cm]
			$\gamma_{1}$ & $-0.0022$ & $0.0064$ & 0.735 & \makecell{Effect Size of \\fully vaccinated percentage}\\
			\addlinespace[0.2cm]
			$\gamma_{2}$ & $-0.0042$ & $0.0037$ & 0.254 & \makecell{Difference in Effect Size of \\fully vaccination percentage between \\ 65+ population and 18-64 population}\\ 
			\bottomrule
		\end{tabular}
		\caption{Negative Binomial Regression with Five Broken Sticks for the Time Trend , Full Vaccination Rate and Interaction between Age Group and Vaccination Rate}\label{sticknbfullvcage}
	\end{table}	
\end{enumerate}

\end{document}
