\documentclass[12pt]{article}
\usepackage{paralist,amsmath,amssymb,thumbpdf,lmodern}
\usepackage{algorithm}
\usepackage{algpseudocode}
\usepackage[margin=0.5in]{geometry}
\usepackage{graphicx}
\usepackage{enumerate}
\usepackage{physics}
\usepackage{amsfonts}
\usepackage{bbm}
\usepackage{xcolor}
\usepackage{subcaption}
\usepackage{graphicx}
\usepackage{multirow}
\usepackage{amsfonts}
\usepackage{booktabs}
\usepackage{diagbox}
\usepackage[outdir=./]{epstopdf}
\usepackage{caption}
\usepackage{bm}
\usepackage{makecell}
\usepackage{ifxetex,ifluatex}
\newcommand*\diff{\mathop{}\!\mathrm{d}}
\begingroup\expandafter\expandafter\expandafter\endgroup
\expandafter\ifx\csname IncludeInRelease\endcsname\relax
  \usepackage{fixltx2e}
\fi
\ifnum 0\ifxetex 1\fi\ifluatex 1\fi=0 % if pdftex
  \usepackage[T1]{fontenc}
  \usepackage[utf8]{inputenc}
  \usepackage{textcomp} % provides euro and other symbols
\else % if luatex or xelatex
  \usepackage{unicode-math}
  \defaultfontfeatures{Ligatures=TeX,Scale=MatchLowercase}
\fi
\usepackage{hyperref}
\hypersetup{
    colorlinks=true,
    linkcolor=blue,
    filecolor=magenta,      
    urlcolor=blue,
}
\urlstyle{same}  % don't use monospace font for urls
\setlength{\emergencystretch}{3em}  % prevent overfull lines
\usepackage{newtxtext,newtxmath}

\title{Biostat 830 HW2}
\date{}
\author{Mukai Wang 98830336}

\makeatletter
\newenvironment{breakablealgorithm}
{% \begin{breakablealgorithm}
		\begin{center}
			\refstepcounter{algorithm}% New algorithm
			\hrule height.8pt depth0pt \kern2pt% \@fs@pre for \@fs@ruled
			\renewcommand{\caption}[2][\relax]{% Make a new \caption
				{\raggedright\textbf{\fname@algorithm~\thealgorithm} ##2\par}%
				\ifx\relax##1\relax % #1 is \relax
				\addcontentsline{loa}{algorithm}{\protect\numberline{\thealgorithm}##2}%
				\else % #1 is not \relax
				\addcontentsline{loa}{algorithm}{\protect\numberline{\thealgorithm}##1}%
				\fi
				\kern2pt\hrule\kern2pt
			}
		}{% \end{breakablealgorithm}
		\kern2pt\hrule\relax% \@fs@post for \@fs@ruled
	\end{center}
}
\makeatother


\begin{document}
\maketitle

\section*{Problem 1}

\begin{enumerate}[(a)]
	\item Since $Y\sim \text{Poisson}(\mu)$ and $\mu \sim \text{Gamma}(\kappa, \theta)$, 
	\begin{align*}
	P\left(\left.Y \right\vert\mu \right)& = \frac{e^{-\mu} \mu^{Y}}{Y!}\\
	\pi\left(\left.\mu \right\vert \kappa, \theta\right) &= \frac{\theta^{\kappa}}{\Gamma(\kappa)}\mu^{\kappa - 1}e^{-\theta\mu}\diff \mu 
	\end{align*}
	Therefore the marginal distribution of $Y$ is
	\begin{align*}
		\pi\left(\left.Y\right\vert\kappa, \theta \right)&= \int_{0}^{\infty} P\left(\left.Y \right\vert\mu \right) \pi\left(\left.\mu \right\vert \kappa, \theta\right)\diff \mu\\
		&=\frac{\theta^{\kappa}}{Y!\cdot \Gamma(\kappa)}\cdot \int_{0}^{\infty}e^{-(1+\theta)\mu} \cdot \mu^{Y+K-1}\diff \mu \\
		&=\frac{\theta^{\kappa} \Gamma(Y+\kappa)}{Y! \cdot \Gamma(\kappa) \left(1+\theta\right)^{Y+\kappa} }\cdot \int_{0}^{\infty} \frac{(1+\theta)^{Y+\kappa}}{\Gamma(Y+\kappa)}e^{-(1+\theta)\mu}\mu^{Y+K-1}\diff \mu\\
		&=\begin{pmatrix}Y+\kappa - 1 \\ Y\end{pmatrix}\left(\frac{1}{1+\theta}\right)^{Y}\left(\frac{\theta}{1+\theta}\right)^{\kappa}
	\end{align*}
	If we consider $\kappa$ to be an integer, then we have $Y \sim \text{NegBin}(\kappa, 1/(1+\theta))$.
	\item Given that the marginal distribution is a negative binomial distribution, then we have the expectation and variance to be
	\begin{align*}
		\mathbb{E}[Y] &= \kappa\theta\\
		\text{Var}(Y) &= \kappa\theta(1+\theta)
	\end{align*}
	\item We can reparameterize the expression s of expectation and variance to be
	\begin{align*}
		\mathbb{E}[Y] &= \mu \\
		\text{Var}(Y) &= \mu(1+\sigma^2 \mu)
	\end{align*}
	in which $\mu = \kappa\theta$ and $\sigma^2 = 1/\kappa$. We can tell that the shape parameter of the gamma distribution dictates the overdispersion. A suitable null hypothesis is $H_{0} : \sigma^2 = 0$
	\item The latent process in the polio example is assumed to follow a stationary gamma process(assuming no seasonal trends). We have learnt that $\kappa$ dictates the variance of gamma distribution, therefore it also controls the involvement of the latent process in the Poisson-gamma hierarchical model.
\end{enumerate}
 

\section*{Problem 2}
\emph{Note: Because the provisional weekly COVID-19 deaths by age and sex are only available on \href{https://data.cdc.gov/NCHS/Provisional-COVID-19-Deaths-by-Week-Sex-and-Age/vsak-wrfu}{national level}, I also referred to the weekly aggregated death counts of Michigan from \href{https://www.michigan.gov/coronavirus/-/media/Project/Websites/coronavirus/Michigan-Data/10-04-2022/Datasets/Cases-and-Deaths-by-County-and-by-Date-of-Symptom-Onset-or-by-Date-of-Death2022-10-04.xlsx?rev=d9568cd19a45423aac6c03af0ce87707&hash=EF2BBF61BD1BFFB0E9F20527E91DCC47}{the state government}. The state level death counts by age groups and sex groups are estimated based on state level totals and national level proportions.}  
\begin{enumerate}[(a)]
	\item  The death counts were provided in 10 age groups(under 1 year, 1-4, 5-14, 15-24, 25-34, 35-44, 45-54, 55-64, 65-74, 75-84, 85+). Before aggregating death counts in three target age groups(< 18, 19-64, 65+), I split the deaths in age group 15-24 into two age groups(15-18, 19-24) in 40/60 proportion. From the weekly death count time series plot(Figure \ref{weekdeath}), I can notice that death events peaked at four time points for people of all age groups and sex groups. They are April to May 2020(beginning of pandemic), December 2020 to January 2021(the first winter), May 2021(the second spring) and December 2021 to January 2022(the second winter). In panel A, I plot the death trends by three age groups. There were virtually no death cases for the youngest age group. Death counts from the oldest age group have been consistently higher than the other two age groups, and were considerably higher during the four severe periods. There hasn't been significant differences between male and female in terms of death cases throughout the past two years. The death counts of males were slightly higher than females during the four severe periods.
	\begin{figure}[htbp]
		\centering
		\includegraphics[scale=0.8]{DeathTrajectory.pdf}
		\caption{Weekly Death Count Trends by Age Group(A) and Sex Group(B)}\label{weekdeath}
	\end{figure} 
	
	\item The vaccination data is provided for four age groups(5-11, 12-17, 18-64, 65+). The emergency use authorization by CDC were given for these different age groups on different dates. Adults between 18-64 began vaccination on 2/13/2021. Elderly people older than 65 began vaccination on 3/5/2021. Adolescents between 12 and 18 began vaccination on 5/13/2021. Children between 5 and 12 began vaccination on 11/3/2021. This background information can explain why the vaccination rates for two older populations(18-64 and 65+) took off earlier than that for the youngest population(under 18) in Figure \ref{vaccinationplot}A. The proportion of population that had at least one dose rose quickly at the beginning and later plateaued for population of all ages. The oldest population had the highest one-dose vaccination rate, followed by adults and the underaged.
	
	\item According to Figure \ref{vaccinationplot}B, the trend for the fully vaccinated is the basically the same as those with at least one dose. The only difference is that the full vaccination rate is slightly lower than the one-dose vaccination rate by about 10\% for all age groups by the end of September 2022.
	
	\begin{figure}[htbp]
		\centering
		\includegraphics[scale=0.8]{vaccination.pdf}
		\caption{The proportion of population with at least one dose of vaccination(A) and the proportion of fully vaccinated population(B)}\label{vaccinationplot}
	\end{figure}
\end{enumerate}


\section*{Problem 3}

\begin{enumerate}[(a)]
	\item We fit a Poisson regression on the weekly death counts across 132 weeks. The Poisson model with log link is parameterized as
	\begin{align*}
		\log(\frac{\mathbb{E}[n_{ik}]}{N_{k}}) &= \beta_{0} + \beta_{1}k + \alpha_{0} t_{i} 
	\end{align*}
	$t_{i} = i$ represents the week number ($i=0,1,\cdots 131$). $k=0,1,2$ represents three age groups(under 18, 19-64 and 65+).
	$n_{ik}$ is the death count of people from age group $k$ in week $t_{i}$. $N_{k}$ represents the total population from age group $k$. According to \href{https://www.census.gov/quickfacts/MI}{census bureau data}, $N_{0}\approx 2.14$ million, $N_{1} \approx 6.05$ million and  $N_{2}\approx 1.81$ million. $\alpha_0$ represents the effect of time trend.  $\beta_1$ represents the difference in death rates between two adjacent age groups. $\beta_{1}$ is the parameter of interest.  $\hat{\beta}_{1} = 2.38$ with a p value smaller than $2\times 10^{-16}$, indicating that age is a significant risk factor. The death rate of population over 65 is about $\exp(2.38) = 10.8$ times the death rate of people between 19 and 64. $\hat{\alpha}_{0} = -2.54\times 10^{-3}$ with a p value smaller than $2\times 10^{-16}$, indicating that the death rate is steadily decreasing week by week, although the death rate is only decreasing at a tiny $0.25\%$ per week.
	\item In the previous question, $\text{deviance}/\text{df}=30783/393=78.3 > 1$, indicating that overdispersion exists in the Poisson regression. We use negative binomial regression instead of Poisson regression, which means the variance $V(n) = \mathbb{E}[n] +\mathbb{E}^{2}[n]/\theta$. Under the new setting, $\hat{\beta}_{1} = 3.1$. The significance conclusion doesn't change but $\text{deviance}/\text{df}=417/393=1.06$ is much smaller than that of Poisson regression. $\hat{\theta}=0.96$. $\hat{\alpha}_{0}$ is not significantly different from zero after fitting negative binomial regression to account for overdispersion, indicating that there isn't significant time trend after all.
	
	\item In this subquestion we set five different slopes for the time trends separated by four time knots. The first time knot is set at December 14, 2020(availability of vaccines to the general public, week 81). The second time knot is set at June 26, 2021(the time when the percent of infection by the delta variant exceeded 50\% of the US population, week 137). The third time knot is set at Nov 29, 2021(the time when the omicron variant began to spread in South Africa, week 181). The fourth time knot is set at May 22, 2022(when the omicron became dominant in the USA, 231). The broken-stick model can be parametrized as
	 \begin{align*}
		\log(\frac{\mathbb{E}[n_{ik}]}{N_{k}}) &= \beta_{0} + \beta_{1}k + \left(\alpha_{0} + \alpha_{1}\mathbb{I}(t_{i} \geq 81) + \alpha_{2}\mathbb{I}(t_{i} \geq 137) +\alpha_{3} \mathbb{I}(t_{i} \geq 181) + \alpha_{4}\mathbb{I}(t_{i} \geq 231)\right) t_{i}
	\end{align*}
	The summary of the Poisson regression coefficients are in the third column of Table \ref{p3table}. The estimated difference of log death rate between adjacent age groups is the same as the previous questions. The death rates were decreasing throughout time, but the decrease rate of log death rate changed at each time knot that we chose. The decrease rate slowed down after Dec 14 2020, sped up after June 26, 2021, slowed down again after Nov 29, 2021 and sped up after May 22, 2022. The summary of the negative binomial regression coefficients are in the fourth column of Table \ref{p3table}. The big difference between negative binomial regression and poisson regression lies in the inference of the slopes of the "broken sticks". After considering overdispersion, the decrease rate of the log death rate stayed relatively unchanged until May 22 2022.
	
	\begin{table}[htbp]
		\centering
		\begin{tabular}{cccccc}
			\toprule
			Coefficient & Poi & NB & Poi w/ BS & NB w/ BS & Meaning\\
			\midrule
			$\beta_{0}$ & $-13.6^\star$ & $-14.9^\star$ & $-13.5^\star$ & $-14.9^\star$ & \makecell{Expected log death rate \\of population younger than 18\\ on 03/14/2020 } \\ 
			\addlinespace[0.2cm]
			$\beta_{1}$ & $2.38^\star$ & $3.11^\star$ & $2.37^\star$ & $3.14^\star$ & \makecell{Difference of log death rate \\between adjacent age groups }\\
			\addlinespace[0.2cm]
			$\alpha_{0}$ & $-2.54\times 10^{-3\star}$ & $-1.48\times 10^{-4}$ & $-0.010^\star$ & $-8.15\times 10^{-3}$ & \makecell{Average decrease per week of \\ log Death Rate }\\
			\addlinespace[0.2cm]
			$\alpha_{1}$ & - & - & $6.78\times 10^{-3\star}$  & $9.73\times 10^{-3}$ & \makecell{Change in decrease rate \\ of log death rate\\ after Dec 14, 2020}\\
			\addlinespace[0.2cm]
			$\alpha_{2}$ & - & - & $-1.80\times 10^{-3\star}$ & $-3.04\times 10^{-4}$ & \makecell{Change in decrease rate \\ of log death rate\\ after June 26, 2021}\\
			\addlinespace[0.2cm]
			$\alpha_{3}$ & - & - & $5.73\times 10^{-3\star}$ & $2.88\times 10^{-3}$ & \makecell{Change in decrease rate \\ of log death rate\\ after Nov 29, 2021}\\
			\addlinespace[0.2cm]
			$\alpha_{4}$ & - & - & $-9.48\times 10^{-3\star}$ & $-0.013^\star$ & \makecell{Change in decrease rate \\ of log death rate\\ after May 22, 2022}\\
			\bottomrule
		\end{tabular}
		\caption{Regression Outcomes for Problem 3. The first column contains the Poisson regression with a constant effect for time trends. The second column contains the negative binomial regression with a constant effect for time trends. The third and fourth column adds a "broken stick" effect on the time trends. Coefficients with $\star$ indicate that the P values are smaller than 0.05.}\label{p3table}
	\end{table}	
	
	
	
	
\end{enumerate}


\section*{Problem 4}

\begin{enumerate}[(a)]
	\item After adding the full vaccination information as the main effect, the model becomes
	\begin{align*}
		\log(\frac{\mathbb{E}[n_{ik}]}{N_{k}}) &= \beta_{0} + \beta_{1}k +\gamma_{0} v_{ik} + \left(\alpha_{0} + \alpha_{1}\mathbb{I}(t_{i} \geq 81) + \alpha_{2}\mathbb{I}(t_{i} \geq 137) +\alpha_{3} \mathbb{I}(t_{i} \geq 181) + \alpha_{4}\mathbb{I}(t_{i} \geq 231)\right) t_{i} 
	\end{align*}
	in which $\gamma_{0}$ is the effect size of the fully vaccination rate. $v_{ik}$ is the fully vaccination percentage for age group $k$ at the end of week $i$. I fit the negative binomial distribution to take into account overdispersion. The coefficients are summarized in the first column of Table \ref{P4table}. The fully vaccination rate didn't significantly reduce the mortality rate. The other coefficients are almost the same as the negative binomial model without vaccination rate(the fourth column of Table \ref{p3table}).
	
	
	\item We add an interaction term between age groups and vaccination to determine if there is significant difference between age groups regarding effectiveness of full vaccination on reducing mortality. The model is
		\begin{align*}
		\log(\frac{\mathbb{E}[n_{ik}]}{N_{k}}) &= \beta_{0} + \beta_{1}k +(\gamma_{0} + \gamma_{1}k) v_{ik} +\\ 
		&\quad \left(\alpha_{0} + \alpha_{1}\mathbb{I}(t_{i} \geq 81) + \alpha_{2}\mathbb{I}(t_{i} \geq 137) +\alpha_{3} \mathbb{I}(t_{i} \geq 181) + \alpha_{4}\mathbb{I}(t_{i} \geq 231)\right) t_{i}
	\end{align*}
	
	There is an extra parameter $\gamma_{1}$ to represent the difference in effect sizes of full vaccination rates between adjacent age groups. The fitted parameters are in the second column of Table \ref{P4table}. We notice that $\hat{\gamma}_{1} = -0.018$ with a P value of $1.25\times 10^{-9}$. This indicates that the vaccination provided stronger protection to older population than younger ones. It's also worth noting that $\gamma_{0}=0.033$ with a P value of $1.11\times 10^{-5}$. The positive value indicates that vaccination had a negative impact on population under age 18. This is a peculiar result, but we shouldn't take it too seriously because of the sparsity of death events in the youngest age group.
	
	\begin{table}[htbp]
		\centering
		\begin{tabular}{cccc}
			\toprule
			Coefficient & \makecell{NB Model \\ Without Interaction} & \makecell{NB Model \\ with Interaction } &  Meaning\\
			\midrule
			$\beta_{0}$ & $-15.0^\star$ & $-15.3^\star$ & \makecell{Expected log death rate \\of population younger than 18\\ on 03/14/2020 }\\
			\addlinespace[0.2cm]
			$\beta_{1}$ & $3.21^\star$ & $3.51^\star$ &  \makecell{Difference of log death rate \\between adjacent age groups}\\
			\addlinespace[0.2cm]
			$\gamma_{0}$ & $-3.78\times 10^{-3}$ & $0.033^\star$ &  \makecell{Effect sizes of \\fully vaccination percentage}\\
			\addlinespace[0.2cm]
			$\gamma_{1}$ & - & $-0.018^\star$ &  \makecell{Difference in effect sizes of \\fully vaccination percentage \\ between adjacent age groups}\\
			\addlinespace[0.2cm] 
			$\alpha_{0}$ & $-7.76\times 10^{-3}$ & $-9.9\times 10^{-3}$ &   \makecell{Average decrease per week of \\ log Death Rate }\\
			\addlinespace[0.2cm]
			$\alpha_{1}$ & $0.011$ & $8.5\times 10^{-3}$ &  \makecell{Change in decrease rate \\ of log death rate\\ after Dec 14, 2020}\\
			\addlinespace[0.2cm]
			$\alpha_{2}$ & $6.44\times 10^{-4}$ & $-3.5\times 10^{-3}$ &  \makecell{Change in decrease rate \\ of log death rate\\ after June 26, 2021}\\
			\addlinespace[0.2cm]
			$\alpha_{3}$ & $2.7\times 10^{-3}$ & $3.2\times 10^{-3}$ & \makecell{Change in decrease rate \\ of log death rate\\ after Nov 29, 2021}\\
			\addlinespace[0.2cm]
			$\alpha_{4}$ & $-0.014^\star$ & $-0.011^\star$ &  \makecell{Change in decrease rate \\ of log death rate\\ after May 22, 2022}\\
			
			\bottomrule
		\end{tabular}
		\caption{Regression Outcomes for Problem 4. The first column contains estimates for the negative binomial regression model with the percent of full vaccination. The negative binomial regression for the second column adds an interaction term between age groups and full vaccination rates. Coefficients with $\star$ indicate that the P values are smaller than 0.05.}\label{P4table}
	\end{table}	
\end{enumerate}

\end{document}
