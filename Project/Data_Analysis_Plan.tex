\documentclass[12pt]{article}
\usepackage{paralist,amsmath,amssymb,thumbpdf,lmodern}
\usepackage{algorithm}
\usepackage{algpseudocode}
\usepackage[margin=0.8in]{geometry}
\usepackage{graphicx}
\usepackage{enumerate}
\usepackage{physics}
\usepackage{amsfonts}
\usepackage{bbm}
\usepackage{xcolor}
\usepackage{subcaption}
\usepackage{graphicx}
\usepackage{multirow}
\usepackage{amsfonts}
\usepackage{booktabs}
\usepackage{diagbox}
\usepackage[outdir=./]{epstopdf}
\usepackage{caption}
\usepackage{bm}
\usepackage{makecell}
\usepackage{ifxetex,ifluatex}
\newcommand*\diff{\mathop{}\!\mathrm{d}}
\begingroup\expandafter\expandafter\expandafter\endgroup
\expandafter\ifx\csname IncludeInRelease\endcsname\relax
  \usepackage{fixltx2e}
\fi
\ifnum 0\ifxetex 1\fi\ifluatex 1\fi=0 % if pdftex
  \usepackage[T1]{fontenc}
  \usepackage[utf8]{inputenc}
  \usepackage{textcomp} % provides euro and other symbols
\else % if luatex or xelatex
  \usepackage{unicode-math}
  \defaultfontfeatures{Ligatures=TeX,Scale=MatchLowercase}
\fi
\usepackage{hyperref}
\hypersetup{
    colorlinks=true,
    linkcolor=blue,
    filecolor=magenta,      
    urlcolor=blue,
}
\urlstyle{same}  % don't use monospace font for urls
\setlength{\emergencystretch}{3em}  % prevent overfull lines
\usepackage{newtxtext,newtxmath}


\title{Data Analysis Plan for BIOSTAT 830 \emph{Infectious Disease Modeling} Project at University of Michigan }
\date{}
\author{Mukai Wang}

\begin{document}
\maketitle

I plan to model the relationship between the severity of the pandemic and people's mental and financial state throughout the past two years. The \href{https://uasdata.usc.edu/index.php}{UAS Los Angeles County survey data} provides longitudinal responses from April 2020 to July 2021 such as 
\begin{itemize}
\item Over the last 7 days, how often have you been bothered by not being able to stop or control worrying?
\item In the past seven days, did you eat less than you thought you should because of a lack of money or other
resources?
\end{itemize}
The responses to these questions are good indicators of mental and financial status. I plan to use compartmental models(e.g. SIR, SEIR) to the pandemic incidence data(\href{https://data.cdc.gov/browse?q=COVID-19&sortBy=relevance}{national} and/or \href{https://data.ca.gov/group/covid-19}{state} level) and use the time varying infection rate/recovery rate to represent the severity of pandemic. When I examine the effect of pandemic severity, I will adjust for the demographic parameters(age, gender etc) collected in the survey.

The data will be stored on \href{https://its.umich.edu/communication/collaboration/dropbox}{Dropbox} provided by University of Michigan. Dropbox complies with the data storage requirement of UAS survey data(\href{https://safecomputing.umich.edu/dataguide/?q=node/251}{link}). The deliverable of this project is currently planned to be presented in class only. I will notify USC Center for Economic and Social Research if I plan to publish the findings to a bigger audience.

\end{document}